\documentclass{beamer}

% Use metropolis theme
\usepackage[progressbar=foot]{theme/beamerthememetropolis}
\usepackage{minted}


\title{Implementing lazy functional languages on stock hardware:}
\subtitle{the Spineless Tagless G-Machine}

\date{\today}
\author{Joe Jevnik}
\institute{Papers We Love - Boston}

\begin{document}
\maketitle

\begin{frame}{Background}
  \begin{block}{Existing abstract machines\ldots}
    \begin{itemize}
    \item[]<2-> are too abstract
    \item[]<3-> don't say enough about data structure traversal
    \item[]<4-> leave unboxed values to the code generator
    \end{itemize}
  \end{block}
\end{frame}

\begin{frame}{Overview}
  \begin{enumerate}
  \item<1-> the design space
  \item<2-> the abstract machine (the STG language)
  \item<3-> mapping the abstract machine onto real hardware
  \end{enumerate}
\end{frame}

\section{Design Space}

\begin{frame}{What is a Functional Language?}
  \begin{itemize}
  \item[]<1-> \begin{definition}[Functional Language]
      A programming language which treats computation as the evaluation of
      mathematical functions.
    \end{definition}
  \item[]<2-> \begin{block}{Examples}
      \begin{itemize}
      \item SML
      \item OCaml
      \item Scala
      \item Haskell*
      \end{itemize}
    \end{block}
  \end{itemize}
\end{frame}

\begin{frame}[fragile]{Features of Functional Languages}
  \begin{onlyenv}<1>
    \begin{block}{Higher order functions}
      \begin{minted}{haskell}
plusOne :: Int -> Int
plusOne x = x + 1

xs :: [Int]
xs = [1, 2, 3]

xsPlusOne :: [Int]
xsPlusOne = map plusOne xs
      \end{minted}
    \end{block}
  \end{onlyenv}

  \begin{onlyenv}<2-3>
    \begin{block}{Algebraic data types}
      \begin{onlyenv}<2>
        \begin{minted}{haskell}
data Maybe a = Nothing | Just a
data List a = Nil | Cons a (List a)
data Tuple3 a b c = MkTuple3 a b c
data Tree a = Leaf a | Branch (Tree a) (Tree a)

safeHead :: [a] -> Maybe a
safeHead = case xs of
    Nil             -> Nothing
    Cons head tail -> Just head
        \end{minted}
      \end{onlyenv}

      \begin{onlyenv}<3>
        \begin{minted}{haskell}
data Bool = False | True

conditional :: Bool -> a -> a
conditional p whenFalse whenTrue = case p of
    False -> whenFalse
    True  -> whenTrue
        \end{minted}
      \end{onlyenv}
    \end{block}
  \end{onlyenv}

  \begin{onlyenv}<4-6>
    \begin{block}{Immutable data}
      \begin{itemize}
      \item[]<4-> \begin{minted}{haskell}
insort value [] = [value]
insort value (x:xs)
    | value < x = value : x : xs
    | otherwise = x : insort value xs
        \end{minted}
      \item[]<5-> \rule{\textwidth}{0.4pt} \begin{minted}{haskell}
> let xs = [1,2,4,5]
> insort 3 xs
[1,2,3,4,5]
        \end{minted}
      \item[]<6-> \begin{minted}{haskell}
> xs
[1,2,4,5]
        \end{minted}
      \end{itemize}
    \end{block}
  \end{onlyenv}

  \begin{onlyenv}<7->
    \begin{block}{Currying}
      \begin{onlyenv}<7-12>
        \begin{itemize}
        \item[]<7-> \begin{minted}{haskell}
> let f x y = x
          \end{minted}
        \item[]<8-> \begin{minted}{haskell}
> :t f
f :: a -> b -> a
          \end{minted}
        \item[]<9-> \begin{minted}{haskell}
f :: a -> (b -> a)
          \end{minted}
        \item[]<10-> \begin{minted}{haskell}
> f 1 2
1
          \end{minted}
        \item[]<11-> \begin{minted}{haskell}
> (f 1) 2
1
          \end{minted}
        \item[]<12-> \begin{minted}{haskell}
> :t f 1
f 1 :: Num a => b -> a
          \end{minted}
        \end{itemize}
      \end{onlyenv}

      \begin{onlyenv}<13->
        \begin{itemize}
        \item[]<13-> \begin{minted}{sml}
- fun f (x, y) = x;
val f = fn : 'a * 'b -> 'a
          \end{minted}
        \item[]<14-> \begin{minted}{sml}
- f (1, 2);
val it = 1 : int
          \end{minted}
        \item[]<15-> \begin{minted}{sml}
- f 1
stdIn:3.1-3.4 Error: operator and operand don't agree [overload conflict]
  operator domain: 'Z * 'Y
  operand:         [int ty]
  in expression:
    f 1
          \end{minted}
        \end{itemize}
      \end{onlyenv}
    \end{block}
  \end{onlyenv}
\end{frame}

\begin{frame}[fragile]{What is Lazy Evaluation?}
  \begin{onlyenv}<1-2>
    \begin{itemize}
    \item[]<1-> \begin{definition}[Lazy Evaluation]
        An evaluation strategy where expressions are evaluated only when their
        result is needed. By default, all expressions result in a deferred
        computation.

        Lazy Evaluation is also known as ``call by need''.
      \end{definition}
    \item[]<2-> \begin{block}{Examples}
        \begin{itemize}
        \item Python generators
        \item Scala \texttt{lazyval}
        \item C++ expression templates
        \end{itemize}
      \end{block}
    \end{itemize}
  \end{onlyenv}

  \begin{onlyenv}<3-6>
    \begin{itemize}
    \item[]<3-> \begin{minted}{haskell}
> let x = 1 : x
      \end{minted}
    \item[]<4-> \begin{minted}{haskell}
> x
[1,1,1,1,1,1,1,1,1,1,...
      \end{minted}
    \item[]<5-> \begin{minted}{haskell}
> let fibs = scanl (+) 1 $ 0 : fibs
      \end{minted}
      % $  this is to fix emacs latex mode
    \item[]<6-> \begin{minted}{haskell}
> fibs
[1,1,2,3,5,8,13,21,34,55,...
      \end{minted}
    \end{itemize}
  \end{onlyenv}

  \begin{onlyenv}<7->
    \begin{itemize}
    \item[]<7-> \begin{minted}{haskell}
> let xs = [0, undefined, undefined]
      \end{minted}
    \item[]<8-> \begin{minted}{haskell}
> head $ map (+ 1) xs
1
      \end{minted}
% $ this fixes latex mode
    \end{itemize}
  \end{onlyenv}
\end{frame}

\begin{frame}{Key Questions}
  \begin{itemize}
  \item<1-> How are function values, data values, and unevaluated
    expressions represented?
  \item<2-> How is function application performed?
  \item<3-> How is case analysis performed on data structures?
  \end{itemize}
\end{frame}

\section{Representing Values}

\begin{frame}{Representing Values}
  \begin{block}{Types of Values}
    \begin{itemize}
    \item Evaluated objects
    \item Currently unevaluated expressions
    \end{itemize}
  \end{block}
\end{frame}

\begin{frame}[fragile]{Representing Values}
  \begin{block}{Function values and data values look the same}
    \begin{minted}{haskell}
compose f g x = f (g x)

gxIsFunction :: Num a => a -> a
gxIsFunction = compose ($ 1) (+)

gxIsNumber :: Num a => a -> a
gxIsNumber = compose (+ 1) (* 2)
    \end{minted}
% $  this fixes latex mode
  \end{block}
\end{frame}

\begin{frame}{Closures}
  \begin{definition}[Closure]
    A function paired with some bound arguments.
  \end{definition}

  \begin{definition}[Free Variable]
    A bound argument to a \textbf{closure}.
  \end{definition}
\end{frame}

\begin{frame}[fragile]{Representing Closures}
  \begin{minted}{c}
typedef struct {
    void* (*code)(void** free_variables,
                  void** arguments);
    void* free_variables[];
} closure;
  \end{minted}
\end{frame}

\begin{frame}{Thunks}
  \begin{definition}[Thunk]
    A \textbf{closure} that represents a value which is currently unevaluated.

    Thunks have no arguments, just \textbf{free variables}.
  \end{definition}
\end{frame}

\begin{frame}[fragile]{Cell Model Thunks}
  \begin{itemize}
  \item[]<1-> \begin{minted}{c}
typedef struct {
    void* (*code)(void** free_variables);
    void* value;
    void* free_variables[];
} cell_model_thunk;
    \end{minted}
  \item[]<2-> \begin{minted}{c}
void* evaluate(cell_model_thunk* thunk) {
    if (!thunk->value) {
        thunk->value = \
            thunk->code(thunk->free_variables);
    }
    return thunk->value;
}
    \end{minted}
  \end{itemize}
\end{frame}

\begin{frame}[fragile]{Self-Updating Thunks}
  \begin{itemize}
  \item[]<1-> \begin{minted}{c}
typedef struct {
    void* (*code)(void** free_variables);
    void* free_variables[];
} self_updating_thunk;
    \end{minted}
  \item[]<2-> \begin{minted}{c}
void* evaluate(self_updating_thunk* thunk) {
    return thunk->code(thunk->free_variables);
}
    \end{minted}
  \end{itemize}
\end{frame}

\begin{frame}{Update Frames}
  \begin{definition}[Update Frame]
    Code injected in a \textbf{self-updating thunk} to update the thunk's
    closure to become the result or an \textbf{indirection} to the result.
  \end{definition}

  \begin{definition}[Indirection]
    A thunk whose \texttt{code} cell points to an identity function and whose
    \texttt{free\_variables} point to a single thunk.

    This is used when a \textbf{self-updating thunk}'s result is larger than
    the thunk itself.
  \end{definition}
\end{frame}

\begin{frame}[fragile]{Update Frames}
  \begin{onlyenv}<1>
    \begin{minted}{c}
typedef struct {
    void* (*code)(void** free_variable);
    void* free_variables[2];
} closure_2;

    \end{minted}
  \end{onlyenv}

  \begin{onlyenv}<2>
    \begin{minted}{c}
void update_small(closure_2* closure) {
    closure_2* result = evaluate_self_updating(closure);

    closure->code = result->code;

    closure->free_variables[0] = \
        result->free_variables[0];

    closure->free_variables[1] = \
        result->free_variables[1];
}
    \end{minted}
  \end{onlyenv}
\end{frame}

\begin{frame}[fragile]{Update Frames}
  \begin{minted}{c}
typedef struct {
    void* (*code)(void** free_variable);
    void* free_variables[2];
} closure_2;
  \end{minted}

  \begin{onlyenv}<1-3>
    \begin{itemize}
    \item[]<1-> \begin{block}{Before Update}
        \texttt{\string{code, free\_var\_1, free\_var\_2\string}}
      \end{block}
    \item[]<2-> \begin{block}{After Updating (small value)}
        \texttt{\string{Cons, head, tail\string}}
      \end{block}
    \item[]<3-> \begin{block}{After Updating (big value)}
        \texttt{\string{indirection\_code, value, NULL\string}} \\
        value: \texttt{\string{mkTuple3, a, b, c\string}}
      \end{block}
    \end{itemize}
  \end{onlyenv}

  \begin{onlyenv}<4->
    \begin{minted}{c}
void* indirection_code(void** free_variables) {
    return free_variables[0];
}

void update_large(closure_2* closure) {
    void* result = evaluate_self_updating(closure);
    closure->code = indirection_code;
    closure->free_variables[0] = result;
    closure->free_variables[1] = NULL;
}
    \end{minted}
  \end{onlyenv}
\end{frame}

\begin{frame}[fragile]{Self-Updating Thunk Tricks}
  \begin{itemize}
  \item[]<1-> \begin{block}{Black Hole}
      A code pointer which throws an exception or aborts the program.
    \end{block}
  \item[]<2-> \begin{minted}{haskell}
let a = b
    b = a
in a
    \end{minted}
  \item[]<3-> \begin{block}{Thread Synchronization}
      A code pointer which queues the current thread of execution to resume when
      the thunk is updated.
    \end{block}
  \end{itemize}
\end{frame}

\section{Function Application}

\begin{frame}[fragile]{Currying}
  \begin{itemize}
  \item[]<1-> \begin{minted}{haskell}
> let f x y = x
    \end{minted}
  \item[]<2-> \begin{minted}{haskell}
> :t f
f :: a -> b -> a
    \end{minted}
  \item[]<3-> \begin{minted}{haskell}
f :: a -> (b -> a)
    \end{minted}
  \item[]<4-> \begin{minted}{haskell}
> f 1 2
1
    \end{minted}
  \item[]<5-> \begin{minted}{haskell}
> (f 1) 2
1
    \end{minted}
  \end{itemize}
\end{frame}

\begin{frame}{Eval/Apply}
  \begin{enumerate}
  \item<1-> Evaluate the function
  \item<2-> Evaluate all of the arguments
  \item<3-> Apply the function to the arguments
  \end{enumerate}
\end{frame}

\begin{frame}{Push/Enter}
  \begin{enumerate}
  \item<1-> Push arguments onto evaluation stack
  \item<2-> Tail call (enter) the function
  \item<3-> The function enters the next function without returning
  \end{enumerate}
\end{frame}

\begin{frame}[fragile]{Example}
  \begin{itemize}
  \item[]<1-> \begin{minted}{haskell}
apply3 f x y z = f x y z
    \end{minted}
  \item[]<2-> \begin{block}{Eval/Apply}
      \begin{minted}{haskell}
((f x) y) z
      \end{minted}
    \end{block}
  \item[]<3-> \begin{block}{Push/Enter}
      \begin{minted}{haskell}
push x, y, z
enter f
      \end{minted}
    \end{block}
  \end{itemize}
\end{frame}

\section{Data Structures}

\begin{frame}[fragile]{Algebraic Data Types}
  \begin{itemize}
  \item[]<1-> \begin{minted}{haskell}
data Boolean = False | True
data List a = Nil | Cons a (List a)
data Tuple3 a b c = MkTuple3 a b c
data Int = MkInt Int#
data Tree a = Leaf a | Branch (Tree a) (Tree a)
    \end{minted}
  \item[]<2-> \begin{minted}{haskell}
case xs of
  Nil            -> nilExpression
  Cons head tail -> consExpression
    \end{minted}
  \end{itemize}
\end{frame}

\begin{frame}{Case Expression}
  \begin{enumerate}
  \item<1-> Evaluate the expression
  \item<2-> Scrutinize the value
  \item<3-> Select the proper alternative
  \end{enumerate}
\end{frame}

\begin{frame}[fragile]{Switch on Tag}
  \begin{onlyenv}<1>
    \begin{minted}{haskell}
data List a = Cons a (List a)
            | Nil
    \end{minted}
  \end{onlyenv}

  \begin{onlyenv}<2>
    \begin{minted}{c}
closure* node;
void* alternative_continuation;
size_t case_tag;
#define CONS_TAG 0
void cons_code(void) {
    case_tag = CONS_TAG;
    jump(alternative_continuation);
}
#define NIL_TAG 1
void nil_code(void) {
    case_tag = NIL_TAG;
    jump(alternative_continuation);
}
    \end{minted}
  \end{onlyenv}

  \begin{onlyenv}<3>
    \begin{block}{Cons values}
      \texttt{\string{cons\_code, head, tail\string}}
    \end{block}
    \begin{block}{Nil values}
      \texttt{\string{nil\_code\string}}
    \end{block}

    \begin{minted}{haskell}
case xs of
    Cons head tail -> consAlternative
    Nil            -> nilAlternative
    \end{minted}
  \end{onlyenv}

  \begin{onlyenv}<4>
    \begin{minted}{c}
void some_list_case_expression(closure* c) {
    alternative_continuation = alternatives;
    node = c;
    jump(c->code);
}

void some_list_case_expression_alternatives() {
    switch(case_tag) {
    case CONS_TAG:
        /* body of `Cons` alternative */
    case NIL_TAG:
        /* body of `Nil` alternative */
    }
}
    \end{minted}
  \end{onlyenv}
\end{frame}

\begin{frame}[fragile]{Vectored Return}
  \begin{onlyenv}<1>
    \begin{minted}{c}
void** return_vector;

void cons_code(closure* free_variables) {
    jump(return_vector[0]);
}

void nil_code(closure* free_variables) {
    jump(return_vector[1]);
}
    \end{minted}
  \end{onlyenv}

  \begin{onlyenv}<2>
    \begin{minted}{c}
void* some_list_return_vector[] = {
    cons_alternative,
    nil_alternative,
};

void some_list_case_expression(closure* c) {
    return_vector = some_list_return_vector;
    node = c;
    jump(c->code);
}
    \end{minted}
  \end{onlyenv}
\end{frame}

\begin{frame}{Summary}
  \begin{itemize}
  \item[]<1-> Uniform representation of closures.
  \item[]<2-> Cheap indirections are available when performing updates.
  \item[]<3-> Exceptional conditions (black holes, concurrency) are handled in a
    uniform way.
  \item[]<4-> Multiple return conventions for constructors are available.
  \end{itemize}
\end{frame}

\section{The STG Language}

\begin{frame}{Semantics}
  \begin{tabular}{ | l | l | }
    \hline
    Construct & Operational reading \\ \hline
    Function application & Tail call \\
    Let expression & Heap allocation \\
    Case expression & Evaluation \\
    Constructor application & Return to continuation \\
    \hline
  \end{tabular}
\end{frame}

\begin{frame}{Characteristics}
  \begin{itemize}
  \item[]<1-> All function and constructor arguments are atoms (variables or
    constants).
  \item[]<2-> All constructor calls are saturated.
  \item[]<3-> Pattern matching is performed only by \texttt{case} expressions.
  \item[]<4-> Unboxed values are supported.
  \end{itemize}
\end{frame}

\begin{frame}[fragile]{Translating Haskell to STG}
  \begin{onlyenv}<1-3>
    \begin{block}{Haskell}
      \begin{itemize}
      \item[]<1-> \begin{minted}{haskell}
map f []          = []
map f (head:tail) = (f head) : (map f tail)
        \end{minted}
      \item[]<2-> \begin{minted}{haskell}
map f xs = case xs of
    []          -> []
    (head:tail) -> f head : map f tail
        \end{minted}
      \item[]<3-> \begin{minted}{haskell}
map f xs = case xs of
    []          -> []
    (head:tail) -> let fHead    = f head
                       mapFTail = map f tail
                   in fHead : mapFTail
        \end{minted}
      \end{itemize}
    \end{block}
  \end{onlyenv}

  \begin{onlyenv}<4>
    \begin{block}{STG}
      \begin{minted}{haskell}
map = {} \n {f,xs} ->
  case xs of
    Nil {} -> Nil {}
    Cons {head,tail} ->
      let fHead    = {f,head} \u {} -> f {head}
          mapFTail = {f,tail} \u {} -> map {f,tail}
      in Cons {fHead,mapFTail}
      \end{minted}
    \end{block}
  \end{onlyenv}
\end{frame}

\begin{frame}[fragile]{General Translation Rules}
  \begin{onlyenv}<1>
    \begin{block}{Replace binary application with multiple application.}
      $$(\ldots((f\ e_1)\  e_2)\  \ldots)\ e_n
      \Rightarrow
      f \{e_1,e_2,\ldots,e_n\}$$
    \end{block}
  \end{onlyenv}

  \begin{onlyenv}<2>
    \begin{block}{Saturate primitive operations and constructors with
        $\eta$-expansion.}
      $$C \{e_1,\ldots,e_n\}
      \Rightarrow
      \lambda y_1 \ldots y_m \rightarrow C \{e_1,\ldots,e_n,y_1,\ldots,y_m\}$$

      where $C$ is a primitive operation or constructor of arity $n + m$.
    \end{block}
  \end{onlyenv}

  \begin{onlyenv}<3>
    \begin{block}{Name every non-atomic function argument, and every
        lambda-form, by adding a \texttt{let} expression.}
      \begin{minted}{haskell}
        f (a + 1) (b * 2) c
      \end{minted}
      \begin{minted}{haskell}
        let arg1 = a + 1
            arg2 = b * 2
        in f arg1 arg2 c
      \end{minted}
    \end{block}
  \end{onlyenv}

  \begin{onlyenv}<4>
    \begin{block}{Convert the right-hand side of each \texttt{let} binding into
        a lambda-form by adding free-variable and update-flag information.}
      \begin{minted}{haskell}
        let arg1 = a + 1
            arg2 = b * 2
        in f arg1 arg2 c
      \end{minted}
      \begin{minted}{haskell}
        let arg1 = {a} \u {} -> + {a,1}
            arg2 = {b} \u {} -> * {b,2}
        in f {arg1,arg2,c}
      \end{minted}
    \end{block}
  \end{onlyenv}
\end{frame}

\begin{frame}{Identifying Free Variables}
  \begin{enumerate}
  \item<1-> It is mentioned in the body of the lambda, and
  \item<2-> it is not an argument, and
  \item<3-> it is not bound at the top level of the program.
  \end{enumerate}
\end{frame}

\begin{frame}[fragile]{Closures and updates}
  \begin{onlyenv}<1>
    \begin{block}{Manifest function}
      A manifest function is a lambda-form with a non-empty argument list.

      Manifest functions do not require updating because they are already in
      head normal form.
    \end{block}
  \end{onlyenv}

  \begin{onlyenv}<2>
    \begin{block}{Partial application}
      A partial application is a lambda-form of the form:

      \begin{minted}{haskell}
        {v1,...,vN} \n {} -> f {v1,...,vN}
      \end{minted}

      where \texttt{f} is a known \textbf{manifest function} which takes more
      than $n$ arguments.

      Partial applications do not require updating because they are already in
      head normal form.
    \end{block}
  \end{onlyenv}

  \begin{onlyenv}<3>
    \begin{block}{Constructors}
      A constructor-closure is a lambda-form of the form

      \begin{minted}{haskell}
        {v1,...,vN} \n {} -> C {v1,...,vN}
      \end{minted}

      where \texttt{C} is a constructor.

      Constructor code always returns to the alternative continuation so it does
      not require updating.
    \end{block}
  \end{onlyenv}

  \begin{onlyenv}<4>
    \begin{block}{Thunks}
      All other lambda-forms are thunks and should have their update flag set
      unless the compiler can prove the thunk will only be evaluated at most
      once.

      \begin{minted}{haskell}
        f = {} \n {p,xs} ->
          let j = {p} \n {} -> factorial {p}
          in case xs of
            Nil {}      -> + {j,1}
            Cons {y,ys} -> + {j,2}
      \end{minted}

      \texttt{j} is evaluated at most once so it doesn't need to be updated.
    \end{block}
  \end{onlyenv}
\end{frame}

\begin{frame}[fragile]{Translating Haskell into STG}
  \begin{itemize}
  \item[]<1-> \begin{minted}{haskell}
map1 f = mf
  where mf []          = []
        mf (head:tail) = (f head) : (mf tail)
    \end{minted}
  \item[]<2-> \begin{minted}{haskell}
map1 = {} \n {f} ->
  letrec mf = {f,mf} \n {xs} ->
    case xs of
      Nil {}           -> Nil {}
      Cons {head,tail} ->
        let fHead    = {f,head} \u {} -> f {head}
            mapFTail = {mf,tail} \u {} -> mf {tail}
        in Cons {fHead,mapFTail}
  in mf
    \end{minted}
  \end{itemize}
\end{frame}

\begin{frame}[fragile]{Standard constructors}
  \begin{itemize}
  \item[]<1-> \begin{block}{Source}
      \begin{minted}{haskell}
global = 1
myList = [global]
      \end{minted}
    \end{block}
  \item[]<2-> \begin{block}{Non-standard constructor}
      \begin{minted}{haskell}
global = {} \n {} -> MkInt 1#
nil = {} \n {} -> Nil {}
myList = {} \n {} -> Cons {global,nil}
      \end{minted}
    \end{block}
  \item[]<3-> \begin{block}{Standard constructor}
      \begin{minted}{haskell}
global = {} \n {} -> MkInt 1#
nil = {} \n {} -> Nil {}
myList = {global,nil} \n {} -> Cons {global,nil}
      \end{minted}
    \end{block}
  \end{itemize}
\end{frame}

\begin{frame}[fragile]{Full laziness}
  \begin{itemize}
  \item[]<1-> \begin{minted}{haskell}
f = {x} \n {y} -> let z = {x} \u {} ez
                  in ef
    \end{minted}
  \item[]<2-> \begin{minted}{haskell}
z = {x} \u {} -> ez
f = {x,z} \n {y} -> ef
    \end{minted}
  \end{itemize}
\end{frame}

\begin{frame}[fragile]{Arithmetic and unboxed Values}
  \begin{itemize}
  \item[]<1-> \begin{minted}{haskell}
data Int = MkInt Int#
    \end{minted}
  \item[]<2-> \begin{minted}{c}
typedef struct {
    void* mkint_entry_code;
    int64_t value;
}
      \end{minted}
    \item[]<3-> \begin{minted}{haskell}
(+) = {} \n {a,b} ->
  case a of
    MkInt {a#} ->
      case b of
        MkInt {b#} -> case +# {a#,b#} of
          sum# -> MkInt sum#
    \end{minted}
  \end{itemize}
\end{frame}

\section{Operational Semantics}

\begin{frame}[fragile]{Operational Semantics}
  \begin{onlyenv}<1-6>
    \begin{block}{State}
      \begin{enumerate}
      \item<1-> the code
      \item<2-> the argument stack (values)
      \item<3-> the return stack (continuations)
      \item<4-> the update stack (update frames)
      \item<5-> the heap (closures)
      \item<6-> the global environment (closures)
      \end{enumerate}
    \end{block}
  \end{onlyenv}

  \begin{onlyenv}<7>
    \begin{block}{Value}
      Either a heap address or a primitive integer value.

      Other kinds of primitives (like float) could be supported but it would
      look exactly like int for the purposes of this discussion.
    \end{block}
  \end{onlyenv}

  \begin{onlyenv}<8>
    \begin{block}{Global environment ($\sigma$)}
      A map from name of each top-level name to the address of a closure.
    \end{block}
  \end{onlyenv}

  \begin{onlyenv}<9>
    \begin{block}{Local environment ($\rho$)}
      A map of each local variable name to the address of a closure.
    \end{block}
  \end{onlyenv}

  \begin{onlyenv}<10>
    \begin{block}{$val\ \rho\ \sigma\ v$}
      If $v$ is a primitive integer, return $Int\ v$.

      If $v$ is a variable, return its \textbf{value} in $\rho$.

      If $v$ is not in $\rho$ return its \textbf{value} in $\sigma$.
    \end{block}
  \end{onlyenv}

  \begin{onlyenv}<11>
    \begin{block}{Code}
      \begin{tabular}{ l p{6cm} }
        $Eval\ e\ \rho$ & Evaluate expression $e$ in the environment
                          $\rho$ and apply the value to the arguments
                          on the argument stack. \\
        $Enter\ a$ & Apply the closure at address $a$ to the arguments
                     on the argument stack. \\
        $ReturnCon\ c\ ws$ & Return the constructor $c$ applied to the
                             values $ws$ to the continuation on the
                             return stack. \\
        $ReturnInt\ k$ & Return the primitive integer $k$ to the
                         continuation on the return stack. \\
      \end{tabular}
    \end{block}
  \end{onlyenv}
\end{frame}

\begin{frame}[fragile]{State Transitions}
  \begin{onlyenv}<1>
    \begin{block}{Initial State}
      \begin{itemize}
      \item all stacks are empty
      \item the heap holds all the globally bound closures
      \item $\sigma$ holds the addresses of the globally bound
      \item $\sigma$ holds the address of a closure called \texttt{main}
      \end{itemize}

      \[Eval\ (main\ \{\})\ \{\}\]
    \end{block}
  \end{onlyenv}

  \begin{onlyenv}<2>
    \begin{block}{Function Application: $Eval\ (f\ xs)\ \rho$}
      \begin{enumerate}
      \item push $val\ \rho\ \sigma\ xs$ onto the argument stack
      \item let $Addr\ a = val\ \rho\ \sigma\ f$
      \item move to $Enter\ a$
      \item[] note: $Enter\ a$ creates a fresh $\rho$
      \end{enumerate}
    \end{block}
  \end{onlyenv}

  \begin{onlyenv}<3>
    \begin{block}{Entering Non-Updatable Closures: $Enter\ a$}
      when there are at least as many arguments on the stack as there are
      arguments to $a$\ldots
      \begin{enumerate}
      \item $a$ must be in the heap
      \item update $\rho$ the free variables in $a$
      \item pop the consumed arguments from the argument stack
      \item update $\rho$ with the popped arguments
      \item move to $Eval\ e\ \rho$ where $e$ is the body of $a$
      \end{enumerate}
    \end{block}
  \end{onlyenv}

  \begin{onlyenv}<4>
    \begin{block}{\texttt{let(rec)} Expressions:
        $Eval\ (\texttt{let}\ x_n = v_n\ \texttt{in}\ e)$}
      \begin{enumerate}
      \item add all of the bound closures to the heap
      \item add all of the entries to $\rho$
      \item move to $Eval\ e\ \rho$
      \end{enumerate}
    \end{block}
  \end{onlyenv}

  \begin{onlyenv}<5>
    \begin{block}{Case Expressions: $Eval\ (\texttt{case}\ e\ \texttt{of}\
        alts)$}
      \begin{enumerate}
      \item push $(alts,\rho)$ onto the return stack
      \item move to $Eval\ e\ \rho$
      \end{enumerate}
    \end{block}
  \end{onlyenv}

  \begin{onlyenv}<6>
    \begin{block}{Constructor Application: $Eval\ (C\ xs)\ \rho$}
      move to $ReturnCon\ C\ (val\ \rho\ \sigma\ xs)$
    \end{block}

    \begin{block}{$ReturnCon\ C\ vs$}
      \begin{enumerate}
      \item add $vs$ to $\rho$
      \item pop $alts$ from the return stack
      \item select the alternative whose constructor matches $C$
      \item let $e$ = the body of the selected alternative
      \item move to $Eval\ e\ \rho$
      \end{enumerate}
    \end{block}
  \end{onlyenv}

  \begin{onlyenv}<7>
    \begin{block}{Unboxed Integers: $Eval\ k$}
      move to $ReturnInt\ k$
    \end{block}

    \begin{block}{$ReturnInt\ k$}
      \begin{enumerate}
      \item pop $alts$ from the return stack
      \item select the alternative whose value matches $k$
      \item move to $Eval\ e$
      \end{enumerate}
    \end{block}

    \begin{block}{Built-In Operations:
        $Eval\ (\texttt{f\# \string{a\#,b\#\string}})$}

      move to $ReturnInt\ (f\#\ a\#\ b\#)$
    \end{block}
  \end{onlyenv}

  \begin{onlyenv}<8>
    \begin{block}{Entering Updatable Closures: $Enter\ a$}
      \begin{enumerate}
      \item push a tuple of the argument stack, the return stack, and $a$ onto
        the update stack
      \item clear the old argument and return stacks
      \item move to $Eval\ e\ \rho$ where $e$ is the body of $a$
      \end{enumerate}
    \end{block}
  \end{onlyenv}

  \begin{onlyenv}<9>
    \begin{block}{$ReturnCon\ C\ vs$ when the return stack is empty}
      \begin{enumerate}
      \item pop the top update frame from the update stack
      \item restore the argument and return stacks from the update frame
      \item overwrite the closure saved in the update frame to have a body of:
        \texttt{C \string{vs\ldots\string}}
      \item move to $ReturnCon\ C vs$
      \end{enumerate}
    \end{block}
  \end{onlyenv}

  \begin{onlyenv}<10>
    \begin{block}{Partial Application: $Enter\ a$}
      when there are at less arguments on the stack than there are arguments to
      $a$\ldots
      \begin{enumerate}
      \item pop the top update frame from the update stack
      \item restore the return stack from the update frame
      \item update the saved closure*
      \item push all of the update frame's argument stack members onto the
        argument stack
      \end{enumerate}
    \end{block}
  \end{onlyenv}

  \begin{onlyenv}<11>
    \begin{block}{Update the saved closure}
      \begin{enumerate}
      \item let \texttt{oldFreeVars} be the current free variables of the
        closure saved in the update frame
      \item let \texttt{newFreeVars} be the old argument stack members
      \item let \texttt{remainingArgs} be the arguments of $a$ which were
        not on the stack
      \item let \texttt{e} be the body of $a$
      \item overwrite the closure saved in the update frame to be:
        \begin{minted}{haskell}
{oldFreeVars ++ newFreeVars} \n {remainingArgs} -> e
        \end{minted}
      \end{enumerate}
    \end{block}
  \end{onlyenv}
\end{frame}

\section{Generating Code from STG}

\begin{frame}{Choices?}
  \begin{block}{Code generation options}
    \begin{itemize}
    \item ANSI C
    \item GNU C\only<2>{*}
    \item machine code
    \end{itemize}
  \end{block}
\end{frame}

\begin{frame}{Jumps}
  \begin{block}{Requirements for code labels}
    \begin{itemize}
    \item name an arbitrary sequence of code
    \item can be manipulated (pushed on a stack, stored in a closure, etc\ldots)
    \item can be jumped to
    \end{itemize}
  \end{block}
\end{frame}

\begin{frame}[fragile]{Tiny Interpreter}
  \begin{onlyenv}<1>
    \begin{minted}{c}

typedef void* (*(*code_label)(void))(void);

void interpreter(code_label continuation) {
    while (1) {
        continuation = continuation();
    }
}
    \end{minted}
  \end{onlyenv}

  \begin{onlyenv}<2->
    \begin{block}{Optimizations}
      \begin{itemize}
      \item[]<2-> eliminate register saves
      \item[]<3-> eliminate the frame pointer (used for C debuggers)
      \item[]<4-> use direct jumps instead of returning continuations:
        \begin{minted}{c}
#define JUMP(cont) return cont
#define JUMP(cont) \
    asm volatile ("jump %P0" : "p" (cont))
        \end{minted}
      \end{itemize}
    \end{block}
  \end{onlyenv}
\end{frame}

\begin{frame}[fragile]{Representing Closures}
  \begin{minted}{c}
typedef struct {
    code_label standard_entry_code;
    code_label evacuation_code;
    code_label scavenge_code;
    /* debugging info, misc info */
} info_table;

typedef struct {
    info_table* info;
    void* closure_free_variables[];
    int64_t primitive_free_variables[];
  };
  \end{minted}
\end{frame}

\begin{frame}{Stacks}
  \begin{enumerate}
  \item argument stack (closures)
  \item return stack (code labels)
  \item update stack (update frames)
  \end{enumerate}

  \begin{onlyenv}<1>
    \begin{block}{One Stack}
      All three stacks work in synchrony so we could put all the values on a
      single stack.

      This stresses the garbage collector because we would need a way to
      identify which values are pointers to closures.
    \end{block}
  \end{onlyenv}

  \begin{onlyenv}<2>
    \begin{block}{Two Stacks}
      Use one stack for pointers to closures and another stack for everything
      else.

      The ``A'' stack is the pointer stack (``A'' for Argument).

      The ``B'' stack is the Basic value stack.
    \end{block}
  \end{onlyenv}
\end{frame}

\begin{frame}{The rest}
  \begin{itemize}
  \item garbage collection
  \item black holes
  \item compiling \texttt{case} expressions
  \item different constructor return conventions
  \item partial application
  \item statically allocated closures
  \item optimizations everywhere
  \item ``gory'' details
  \end{itemize}
\end{frame}

\begin{frame}{Key Ideas}
  \begin{itemize}
  \item<1-> \texttt{case} expressions are the only way to drive computation
  \item<2-> all information about the shape and type of an object are controlled
    through the entry code
  \item<3-> control flow moves from \texttt{case} alternative to \texttt{case}
    alternative
  \item<4-> data structures are often synthetic
  \end{itemize}
\end{frame}

\end{document}
